% --------------------------------------------------
% Resumo e abstract (obrigatórios)
% --------------------------------------------------
\resumo{%
  Este trabalho é um breve modelo de trabalho de conclusão de curso utilizando o ambiente Latex.
  Para a confecção deste modelo foi utilizado o pacote de classes \textit{ABNTex} que segue as normas da Associação Brasileira de Normas Técnicas.
  A elaboração de uma monografia pode ser feita sobrescrevendo o conteúdo deste modelo.
}
\palavraschave{Trabalho de Conclusão de Curso. Latex. Monografia.}


% --------------------------------------------------
% Keywords e abstract
% --------------------------------------------------
\abstract{%
This work is a brief model of course completion work using the Latex environment.
For the preparation of this model was used the package of classes \textit{ABNTex} that follows the norms of the Brazilian Association of Technical Norms.
The elaboration of a monography can be done by overwriting the content of this model.
}
\keywords{Course Completion Work. Latex. Monography.}

% Dedicatória (opcional)
\textodedicatoria{%
À minha esposa e aos meus filhos.

Aos meus pais e à minha irmã.
}

% Agradecimentos (opcional)
\textoagradecimentos{%
Agradeço a todos que contribuíram para a realização deste trabalho.
}

% Epígrafe (opcional)
\textoepigrafe{%
    ``As invenções são, sobretudo, o resultado de um trabalho teimoso.''\\
    (Santos Dumont)
}

% Lista de siglas (opcional)
\listasiglas{%
 \begin{itemize}[]
  \item[ABNT] -- Associação Brasileira de Normas Técnicas
  \item[IFMG] -- Instituto Federal de Educação, Ciência e Tecnologia de Minas Gerais
  \item[SQL] -- \textit{Structured Query Language}
  \item[TCC] -- Trabalho de conclusão de curso
 \end{itemize}
}

% Lista de símbolos (opcional)
\listasimbolos{%
 \begin{itemize}[]
   \item[$\mathbb{X}$] -- Variável X
   \item[$\mathsf{I\!R}$] -- Conjunto dos números reais
 \end{itemize}
}
