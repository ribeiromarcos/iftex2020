Este capítulo apresenta alguns modelos de referências \cite{ifmg:2020:manual}.

\section{Livro e/ou Folheto}

Os elementos essenciais são: autor(es), título, subtítulo (se houver), edição, local, editora e data de publicação.
Alguns exemplos:

\vspace*{1em}

\begin{minted}{bib}
@book{chiavenato:2014,
  author    = {Idalberto Chiavenato},
  title     = {Administração},
  subtitle  = {teoria, processo e prática},
  edition   = {5},
  address   = {Barueri},
  publisher = {Manole},
  year      = {2014},
  note      = {\textit{E-book}.},
}
\end{minted}

\noindent
\fullcite{chiavenato:2014}.

\vspace*{1em}

\begin{minted}{bib}
@book{fazio:2011,
  author    = {Fazio, Michael W. and
               Moffett, Marian and
               Wodehouse, Lawrence},
  title     = {A história da arquitetura mundial},
  edition   = {3},
  address   = {Porto Alegre},
  publisher = {AMGH},
  year      = {2011},
}
\end{minted}

\noindent
\fullcite{fazio:2011}.

\section{Trabalho acadêmico}

São considerados trabalhos acadêmicos: trabalho de conclusão de curso, dissertações, teses e outros trabalhos acadêmicos considerados no todo.
Os itens essenciais são: autor(es), título, subtítulo (se houver), ano de depósito, tipo do trabalho (tese, dissertação, trabalho de conclusão de curso e outros), grau (graduação, especialização, mestrado ou doutorado) seguido do curso entre parênteses, vinculação acadêmica e data de apresentação ou defesa.
Alguns exemplos:

\vspace*{1em}

\begin{minted}{bib}
@thesis{oliveira:2016:app_fruta,
  title       = {Desenvolvimento de um aplicativo em plataforma {Android} para auxílio no ensino de {Fruticultura}},
  author      = {Oliveira, Bruno Alberto Soares and
Silva, Gabriel da},
  type        = {Relatório Final de Projeto de Iniciação Científica (Graduação em Engenharia de Computação)},
  institution = {Instituto Federal de Educação, Ciência e Tecnologia de Minas Gerais (IFMG)},
  location    = {Bambuí},
  eventyear   = {2016},
  year        = {2016},
}
\end{minted}

\noindent
\fullcite{oliveira:2016:app_fruta}.

\vspace*{1em}

\begin{minted}{bib}
@thesis{vieira:2020:cpresql,
  title       = {Novo modelo de hierarquia de preferências em consultas com preferências condicionais},
  author      = {Vieira, Lucas Mariano},
  type        = {Trabalho de Conclusão de Curso (Graduação em Engenharia de Computação)},
  institution = {Instituto Federal de Educação, Ciência e Tecnologia de Minas Gerais (IFMG)},
  location    = {Bambuí},
  eventyear   = {2020},
  year        = {2020},
}
\end{minted}

\noindent
\fullcite{vieira:2020:cpresql}.

\vspace*{1em}

\begin{minted}{bib}
@thesis{nascimento:2001,
  author      = {Suzy Regina Nascimento},
  title       = {Oscilações no desempenho de motoristas profissionais, motoristas pluriacidentados e não-motoristas em tarefas de atenção mantida},
  type        = {Dissertação (Mestrado em Psicologia)},
  institution = {Instituto de Psicologia, Universidade de
São Paulo (USP)},
  location    = {São Paulo},
  eventyear   = {2001},
  year        = {2001},
}
\end{minted}

\noindent
\fullcite{nascimento:2001}.

\vspace*{1em}

\begin{minted}{bib}
@thesis{ribeiro:2018,
  author      = {Marcos Roberto Ribeiro},
  title       = {{StreamPref}},
  subtitle    = {Uma linguagem de consulta para dados em fluxo com suporte a preferências},
  type        = {Tese (Doutorado em Ciência da Computação)},
  institution = {Faculdade de Computação, Universidade Federal de Uberlândia (UFU)},
  location    = {Uberlândia},
  eventyear   = {2018},
  year        = {2018},
}
\end{minted}

\noindent
\fullcite{ribeiro:2018}.

\section{Parte de trabalho}

Inclui capítulo, volume, fragmento e outras partes de uma obra, com autor(es) e/ou título próprios.
Os elementos essenciais são: autor(es), título da parte, seguidos da expressão ``In:'', e da referência completa da monografia no todo. No final da referência, deve-se informar a descrição física da parte.
Exemplo:

\vspace*{1em}

\begin{minted}{bib}
@incollection{martins:2015,
  author     = {José Rodolfo S. Martins},
  title      = {Obras de macrodrenagem},
  pages      = {167--240},
  booktitle  = {Drenagem urbana},
  editor     = {TUCCI, Carlos E. M. Tucci and
                Rubem La Laina P. Porto and
                Mário T. Barros},
  editortype = {org.},
  publisher  = {ABRH},
  location   = {Porto Alegre},
  year       = {2015},
}
\end{minted}

\noindent
\fullcite{martins:2015}.

\section{Periódicos}

Nas referências a periódicos como todo, os elementos essenciais são: título, subtítulo (se houver), local de publicação, editora, datas de início e de encerramento da publicação (se houver), e ISSN (se houver).
Exemplo:

\vspace*{1em}

\begin{minted}{bib}
@article{techne:1993,
  title     = {TÉCHNE},
  subtitle  = {revista de tecnologia da construção},
  year      = {1993-},
  location  = {São Paulo},
  publisher = {Pini},
  issn      = {0104-1053},
}
\end{minted}

\noindent
\fullcite{techne:1993}.

\vspace*{1em}

Para os artigos de periódico, os elementos essenciais são: autor (es), título do artigo ou da matéria, subtítulo (se houver), título do periódico, subtítulo (se houver), local de publicação, numeração do volume e/ou ano, número e/ou edição, tomo (se houver), páginas inicial e final, e data ou período de publicação.
Exemplo:

\vspace*{1em}

\begin{minted}{bib}
@article{lelis:2004,
  author    = {Lelis, V. G. and
               Costa, E. D. and
               Ramos, L. P. and
               Alvarenga, L. M. and
               Minim, V. P. R. M.},
  title      = {Aceitabilidade sensorial de doce de leite de diferentes sabores},
  journal    = {Revista do Instituto de Laticínios Cândido Tostes},
  volume     = {59},
  number     = {339},
  pages      = {324-327},
  month      = {jan},
  year       = {2004},
  location   = {Juiz de Fora},
}
\end{minted}

\noindent
\fullcite{lelis:2004}.

\vspace*{1em}

No caso de artigo ou matéria de jornal, os elementos essenciais são: autor(es), título do artigo, subtítulo (se houver), título do jornal, subtítulo de jornal (se houver), local de publicação, numeração do ano e/ou volume, número, data de publicação, seção, caderno ou parte do jornal e a paginação correspondente.
Exemplo:

\vspace*{1em}

\begin{minted}{bib}
@article{naves:1999,
  author       = {P. Naves},
  title        = {Lagos andinos dão banho de beleza},
  journaltitle = {Folha de S. Paulo},
  location     = {São Paulo},
  date         = {1999-06-28},
  note         = {Folha Turismo, Caderno 8, p. 13},
}
\end{minted}

\noindent
\fullcite{naves:1999}.

\section{Evento}

Um evento é o resultado de trabalhos publicados em congressos, seminários, simpósios, encontros, semanas, etc.

Nas referências a um evento como todo, os elementos essenciais são: nome do evento, numeração (se houver), ano e local (cidade) de realização, título do documento, seguidos dos dados de local, editora e data da publicação.
Exemplo:

\vspace*{1em}


\begin{minted}{bib}
@proceedings{labid,
  eventtitle = {Congresso Latino-Americano de Biblioteconomia e Documentação},
  number     = {1},
  venue      = {Salvador},
  eventyear  = {1980},
  title      = {Anais [...]},
  publisher  = {FEBAB},
  address    = {Salvador},
  year       = {1980},
  pagetotal  = {350},
}
\end{minted}

\noindent
\fullcite{labid}.

\vspace*{1em}

Para trabalhos publicados em eventos, Os elementos essenciais são: autor, título do trabalho, seguidos da expressão In: nome do evento, numeração do evento (se houver), ano e local (cidade) de realização, título do documento, local, editora e data da publicação e páginas inicial e final da parte referenciada.
Exemplo:

\vspace*{1em}

\begin{minted}{bib}
@inproceedings{brayner:1994,
  author     = {Brayner, A. R. A. and
                Medeiros, C. B.},
  title      = {Incorporação do tempo em SGBD orientado a objetos},
  eventtitle = {Simpósio Brasileiro de Banco de Dados (SBBD)},
  number     = {IX},
  venue      = {São Paulo},
  eventyear  = {1994},
  booktitle  = {Anais [...]},
  publisher  = {USP},
  address    = {São Paulo},
  year       = {1994},
  pages      = {16--29},
}
\end{minted}

\noindent
\fullcite{brayner:1994}.

% \vspace*{1em}

\section{Patente}

Os elementos essenciais de patentes são: inventor (autor), título, nomes do depositante ou titular e do procurador (se houver), número da patente, data de depósito e data de concessão da patente (se houver).
Exemplo:

\vspace*{1em}

\begin{minted}{bib}
@patent{bertazzoli:2006,
  author     = {Bertazzoli, Rodnei and
                     Silva, João and
                     Mendes, José and
                     Carvalho, Maria},
  title      = {Eletrodos de difusão gasosa modifi cados com catalisadores redox, processo e reator eletroquímico de síntese de peróxido de hidrogênio utilizando os mesmos},
  titleaddon = {Depositante: Universidade Estadual de Campinas. Procurador: Maria Cristina Valim Lourenço Gomes},
  number     = {BR n. PI0600460-1A},
  note       = {Depósito: 27 jan. 2006. Concessão: 25 mar. 2008},
}
\end{minted}

\noindent
\fullcite{bertazzoli:2006}.

\section{Legislação}

Referências a legislações incluem Constituição, Decreto, Decreto-Lei, Emenda Constitucional, Emenda à Lei Orgânica, Lei Complementar, Lei Delegada, Lei Ordinária e Medida Provisória, entre outros.
Os elementos essenciais são: jurisdição, ou cabeçalho da entidade, em letras maiúsculas; epígrafe e ementa transcrita conforme publicada; dados da publicação.

Quando necessário, acrescentam-se à referência os elementos complementares para melhor identificar o documento, como: retificações, alterações, revogações, projetos de origem,
autoria do projeto, dados referentes ao controle de constitucionalidade, vigência, eficácia, consolidação ou atualização.
Em epígrafes e ementas demasiadamente longas, pode-se suprimir parte do texto, desde que não seja alterado o sentido. A supressão deve ser indicada por reticências, entre colchete.
Alguns exemplos:

\vspace*{1em}

\begin{minted}{bib}
@legislation{brasil2002,
  author     = {{Brasil}},
  nameaddon  = {[Constituição (1988)]},
  title      = {Constituição da República Federativa do Brasil},
  titleaddon = {Organizado por Cláudio Brandão de Oliveira},
  location   = {Rio de Janeiro},
  publisher  = {Roma Victor},
  year       = {2002},
}
\end{minted}

\noindent
\fullcite{brasil2002}.

\vspace*{1em}

\begin{minted}{bib}
@legislation{curitiba2007,
  author     = {{Curitiba}},
  title      = {Lei n. 12.092, de 21 de dezembro de 2006},
  titleaddon = {Estima a receita e fixa a despesa do município de curitiba para o exercício financeiro de 2007},
  location   = {Curitiba},
  publisher  = {Câmara Municipal},
  year       = {[2007]},
  url        = {http://dominio.cmc.pr.gov.br/contlei.nsf/l12092-2006},
  urldate    = {2007-03-22},
}
\end{minted}

\noindent
\fullcite{curitiba2007}.

\section{Documento cartográfico}

Documentos cartográficos incluem atlas, mapa, globo, fotografia aérea, entre outros.
Elementos essenciais: autor(es), título, subtítulo (se houver), local, editora, data de publicação, descrição física e escala (se houver). Quando necessário, acrescentam-se elementos complementares à referência para melhor identificar o documento.
Exemplo:

\vspace*{1em}

\begin{minted}{bib}
@image{brasil1979,
  author    = {{Brasil}},
  nameaddon = {Ministério da Marinha},
  title     = {Brasil - costa leste},
  subtitle  = {do Rio Itatiti a Ilhéus},
  edition   = {3},
  location  = {Rio de Janeiro},
  year      = {1979},
  note      = {Carta náutica, N. 1.100. Escala natural 1: 308.541 na lat. 13° 23,50'.},
}
\end{minted}

\noindent
\fullcite{brasil1979}.


\section{Meio eletrônico}

Para informações de acesso exclusivo por meio eletrônico, os elementos essenciais: autor(es), título da informação ou serviço ou produto, versão ou edição (se houver), local, data e descrição física do meio eletrônico. Informações sobre o endereço eletrônico, precedido da expressão ``Disponível em:'' e a data de acesso ao documento, precedida da expressão ``Acesso em:''.

Os demais tipos referências em meio eletrônico devem obedecer aos padrões já especificados, acrescidas das informações relativas à descrição física do meio eletrônico e a data de acesso.


Alguns exemplos de referências em meios eletrônicos:

\vspace*{1em}

\begin{minted}{bib}
@online{nourau,
  title      = {NOU-Rau},
  titleaddon = {software livre},
  version    = {Beta 2},
  location   = {Campinas},
  publisher  = {UNICAMP},
  year       = {2002},
  url        = {www.rau-tu.unicamp.br/nou-rau},
  urldate    = {2002-04-23},
}
\end{minted}

\noindent
\fullcite{nourau}.

\vspace*{1em}

\begin{minted}{bib}
@book{galt:2017,
  author    = {Christopher Galt},
  title     = {O terceiro testamento},
  address   = {São Paulo},
  publisher = {Jangada},
  year      = {2017},
  url       = {http://le-livros.com/wp-content/uploads/2018/10/O-Terceiro-Testamento-Christopher-
-Galt.pdf},
 urldate    = {2018-11-29},
}
\end{minted}

\noindent
\fullcite{galt:2017}.

\vspace*{1em}

\begin{minted}{bib}
@thesis{freitas:2006,
  author      = {Freitas, Daniel Medeiros de},
  title       = {Aproximações entre arquitetura e urbanismo nas intervenções realizadas no hipercentro de Belo Horizonte},
  type        = {Dissertação (Mes-
trado em Arquitetura)},
  institution = {Escola de Arquitetura, Universidade Federal de Minas Gerais (UFMG)},
  location    = {Belo Horizonte},
  eventyear   = {2006},
  year        = {2006},
  url         = {http://hdl.handle.net/1843/RAAO-6VZG2H},
  urldate     = {2018-07-07},
}
\end{minted}

\noindent
\fullcite{freitas:2006}.

\vspace*{1em}

\begin{minted}{bib}
@incollection{carvalho:2009,
  author       = {Carvalho, R. F. de and
                  Marar, J. F},
  title        = {Arquitetura de informação},
  pages        = {169--178},
  booktitle    = {Design e planejamento:},
  booksubtitle = {aspectos tecnológicos},
  editor = {Menezes, Marizilda dos S. and
            Paschoarelli, Luiz C.},
  editortype   = {org.},
  publisher    = {UNESP},
  location     = {São Paulo},
  year         = {2009},
  url          = {http://books.scielo.org/id/mw22b},
  urldate      = {2018-07-06},
}
\end{minted}

\noindent
\fullcite{carvalho:2009}.

\vspace*{1em}

\begin{minted}{bib}
@article{tragante:2018,
  author    = {Tragante, Cinthia Aparecida},
  title      = {A habitação na literatura: as casas nos romances de Machado de Assis e de Lima Barreto},
  journaltitle    = {Risco},
  journalsubtitle = {Revista de Pesquisa em Arquitetura e Urbanismo},
  volume     = {16},
  number     = {1},
  pages      = {10--21},
  year       = {2018},
  location   = {São Paulo},
  url        = {https://www.revistas.usp.br/risco/article/view/125235},
  urldate = {2018-07-06}
}
\end{minted}

\noindent
\fullcite{tragante:2018}.

\vspace*{1em}

\begin{minted}{bib}
@article{fernandes:2018,
  author       = {Fernandes, A. and
                  Cunha, J. P.},
  title        = {Embraer não resistiria sozinha, diz especialista},
  journaltitle = {Folha de S. Paulo},
  date         = {2018-07-06},
  note         = {Mercado},
  location     = {São Paulo},
  url          = {https://www1.folha.uol.com.br/mercado/2018/07/embraer-nao-resistiria-sozinha-diz-es-pecialista.shtml},
  urldate      = {2018-07-06}
}
\end{minted}

\noindent
\fullcite{fernandes:2018}.

\vspace*{1em}

\begin{minted}{bib}
@proceedings{icufpe,
  eventtitle = {Congresso de Iniciação Científica da UFPE},
  number     = {4},
  venue      = {Recife},
  eventyear  = {1996},
  title      = {Anais eletrônicos [...]},
  publisher  = {UFPE},
  address    = {Recife},
  year       = {1996},
  url        = {http://www.propesq.ufpe.br/anais/anais.htm},
  urldate    = {1997-01-21},
}
\end{minted}

\noindent
\fullcite{icufpe}.

\vspace*{1em}

\begin{minted}{bib}
@inproceedings{ribeiro:2017:tpref,
  author     = {Ribeiro, Marcos Roberto and
                Barioni, Maria Camila N. and
                de Amo, Sandra and
                Roncancio, Claudia and
                Labbé, Cyril},
  title      = {Reasoning with temporal preferences over data streams},
  eventtitle = {International Florida Artificial Intelligence Research Society Conference (FLAIRS)},
  number     = {XXX},
  venue      = {Marco Island},
  eventyear  = {2017},
  booktitle  = {Proceedings [...]},
  publisher  = {AAAI Publications},
  address    = {Palo Alto},
  year       = {2017},
  pages      = {700--705},
  url        = {https://www.aaai.org/ocs/index.php/FLAIRS/FLAIRS17/paper/view/15398},
  urldate    = {2023-04-12},
}
\end{minted}

\noindent
\fullcite{ribeiro:2017:tpref}.

